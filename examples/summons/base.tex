%\newcommand{\dagordning}{true}
\documentclass{dTeX-summon}
\TheListKindIs{Preliminär föredragningslista}

% Här skriver du in information om när mötet skall hållas samt vilket
% mötesnummer det är:
% -------------------------------------------------------------------
\TheDocumentNameIs{styrelsemöte}
\TheDateIs{2018-05-02} % Vilket datum
\TheMeetingNumberIs{1} % Vilket nummer i ordningen
\TheFiscalYearIs{2018/2019} % Vilket verksamhetsår
\TheStartTimeIs{12:15} % Vilken tid mötet ska starta
\TheLocationIs{Styretrummet} % Vart mötet ska hållas

% Vem fan är du?
\newcommand{\vemardu}{Sekt Ordf} % ditt namn
\newcommand{\titel}{Ordförande}

% Om du vill skriva in ett datum och inte orkar ta reda på vilken
% veckodag det är kan du använda följande kommando:
% \somedate{yyyy}{mm}{dd}
% Det kommer att skriva ut datumet på ett kompetent sätt.
% -------------------------------------------------------------------

\begin{summon}
  \pagestyle{DTek-meeting}
  \maketitle

  \begin{Agenda}
    \Item{Preliminärer}
      \SubItem{Mötets öppnande}
      \SubItem{Godkännande av föredragningslistan}
        \secr{flytta ut rykten och skvaller}
      \SubItem{Val av justerare}
      \SubItem{Föregående mötesprotokoll}
      \SubItem{Adjungeringar}
        \secr{%
          Medlemmar av styrelsen (kommittéordföranden och presidiet) och
          revisorerna har rätt att närvara, yttra sig, lägga förslag samt rösta
          på styrelsemöten. Om vi vill ge någon annan de tre förstnämnda (exv
          representant för från förening) så adjungerar vi in dessa.
        }

    \Item{Uppföljning av beslut}
      \elab{Foajéstäd och beslut från förra mötet.}
      \secr{Foajén, städ av styrelserum }

    %------------------------------------Fasta punkter

    \Item{Rapporter från möten}
      \elab{Rapport från utskottsmöten och andra relevanta forum. }

    \Item{Röda kort}
      \elab{Har några röda kort städats av och ska vi dela ut några nya? }

    \Item{Bordlagda ärenden}

    \Item{Basenpax}
      \SubItem{11/5 - Frukostklubben}

    %---------------------------------Varierande punkter

    \Item{Inval av D-Lat}
      \secr{Delat ansvar, inget ansvar}
    \Item{Sektionsmöte imorgon}
      \elab{Har vi något vi vill ta upp eller bemöta?}

    \Item{Kläder till styret}
      \secr{%
        Det har lyfts att vi kanske borde ha orangea hoodies för att synas
        bättre. En nackdel är att SouthWest inte tillverkar orangea hoodies. En
        annan nackdel är att vi syns.
      }

    \Item{Hänt i veckan}
      \elab{Ordföranden rapporterar om hur verksamheten går.}
      \secr{Har ni kommit igång med mail osv?}


    \Item{Övriga frågor}


    \Item{Rykten och skvaller}
      \secr{flytta ut!}

    \Item{Mötets avslutande}

  \end{Agenda}
  % ------------------------------------------------------------

  \textit{\vemardu}\\
  \textit{\titel}
\end{summon}
