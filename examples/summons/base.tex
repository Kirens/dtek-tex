%\newcommand{\dagordning}{true}
\documentclass{dTeX-summon}
\dokumenttyp{styrelsemöte}
\listtyp{Preliminär föredragningslista}

% Här skriver du in information om när mötet skall hållas samt vilket
% mötesnummer det är:
% -------------------------------------------------------------------
\title{\typ{}}
\date{1 maj 2018} % Vilket datum
\motesnummer{1} % Vilket nummer i ordningen
\verksamhetsar{2018/2019} % Vilket verksamhetsår
\start{12:15} % Vilken tid mötet ska starta
\plats{Styretrummet} % Vart mötet ska hållas

\newcommand{\ye}{2018} % Vilket år mötet ska hållas
\newcommand{\mo}{05} % Vilken månad mötet ska hållas
\newcommand{\da}{02} % Vilken dag mötet ska hållas

% Vem fan är du?
\newcommand{\vemardu}{Sekt Ordf} % ditt namn
\newcommand{\titel}{Ordförande}

% Om du vill skriva in ett datum och inte orkar ta reda på vilken
% veckodag det är kan du använda följande kommando:
% \somedate{yyyy}{mm}{dd}
% Det kommer att skriva ut datumet på ett kompetent sätt.
% -------------------------------------------------------------------

\begin{document}
  \pagestyle{DTek-meeting}
  \maketitle

  \begin{foredragningslista}
    \punkt{Preliminärer}
      \begin{underpunkter}
        \item Mötets öppnande
        \item Godkännande av föredragningslistan
          \secr{flytta ut rykten och skvaller}
        \item Val av justerare
        \item Föregående mötesprotokoll
        \item Adjungeringar
          \secr{%
            Medlemmar av styrelsen (kommittéordföranden och presidiet) och
            revisorerna har rätt att närvara, yttra sig, lägga förslag samt rösta
            på styrelsemöten. Om vi vill ge någon annan de tre förstnämnda (exv
            representant för från förening) så adjungerar vi in dessa.
          }
      \end{underpunkter}

    \punkt{Uppföljning av beslut}
      \elab{Foajéstäd och beslut från förra mötet.}
      \secr{Foajén, städ av styrelserum }

    %------------------------------------Fasta punkter

    \punkt{Rapporter från möten}
      \elab{Rapport från utskottsmöten och andra relevanta forum. }

    \punkt{Röda kort}
      \elab{Har några röda kort städats av och ska vi dela ut några nya? }

    \punkt{Bordlagda ärenden}

    \punkt{Basenpax}
      \begin{underpunkter}
        \item 11/5 - Frukostklubben
      \end{underpunkter}

    %---------------------------------Varierande punkter

    \punkt{Inval av D-Lat}
      \secr{Delat ansvar, inget ansvar}
    \punkt{Sektionsmöte imorgon}
      \elab{Har vi något vi vill ta upp eller bemöta?}

    \punkt{Kläder till styret}
      \secr{%
        Det har lyfts att vi kanske borde ha orangea hoodies för att synas
        bättre. En nackdel är att SouthWest inte tillverkar orangea hoodies. En
        annan nackdel är att vi syns.
      }

    \punkt{Hänt i veckan}
      \elab{Ordföranden rapporterar om hur verksamheten går.}
      \secr{Har ni kommit igång med mail osv?}


    \punkt{Övriga frågor}


    \punkt{Rykten och skvaller}
      \secr{flytta ut!}

    \punkt{Mötets avslutande}

  \end{foredragningslista}
  % ------------------------------------------------------------

  \textit{\vemardu}\\
  \textit{\titel}
\end{document}
