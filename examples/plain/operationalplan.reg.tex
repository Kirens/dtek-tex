\documentclass{dTeX-base}
\usepackage{titlesec}
\titlespacing*{\section}{0pt}{*3}{*0}

\title{Proposition: Verksamhetsplan 2018/2019}
\TheSubtitleIs{Slutgilltig förslag till plan för sektionsstyrelsen verksamhetsåret}
\author{}
\TheDateIs{7 juli 2019}

\newcommand{\skall}{Sektionsstyrelsen 2018/2019 skall}
\NewDocumentEnvironment{FokusPunkt}{mg}{
  \section*{#1}
  \IfNoValueT{\textit{#2}}

  Sektionsstyrelsen 2018/2019 skall

  \begin{itemize}
}{
  \end{itemize}
}

\begin{document}
  \pagestyle{DTek}
  \maketitle

  \begin{FokusPunkt}{Styrelseform}{%
    Under verksamhetsåret 13/14 utökades presidiet med två ledamöter. Sedan
    dess har den ena posten utvecklats till SAMO och under 17/18 förtydligades
    även det i sektionens styrdokument. Den andra postens syfte var oklart och
    togs under 16/17 bort. Under 17/18 upplevdes det vara ett bra beslut, men
    mer utvärdering behövs.
  }
    \item Utvärdera huruvida det var ett bra beslut att ta bort den sjätte
      posten i presidiet.
  \end{FokusPunkt}

  \begin{FokusPunkt}{Fortsätta arbetet med sektionslokalen}{
    Under de senaste åren så har mycket arbete lagts på Basen och att fortsätta
    det arbetet ligger i sektionens intresse. Under 17/18 avslutades
    köksrenoveringen och det börjades tas fram en plan för renovering av puben.
    Sen kontanterna helt försvunnit från sektionen 17/18 har inbrotten i Basen
    minskat men stölder förkommer ännu.
  }
    \item Fortsätta arbeta med Basen och öka trivseln i lokalen.
    \item Arbeta med renoveringen av Basen.
    \item Följa upp om insatser mot inbrott och stölder i sektionens lokaler
      haft resultat.
  \end{FokusPunkt}

  \begin{FokusPunkt}{Öka engagemanget inom sektionen}{
    Sektionens medlemmar engagerar sig inte i sektionens verksamhet i lika stor
    utsträckning som tidigare. Det har varit många fyllnadsval det senaste och
    sektionsstyrelsen 17/18 anser att det finns stora möjligheter att förändra
    detta till det bättre.
  }
    \item Synliggöra sektionens verksamhet.
    \item Undersöka varför sektionsmedlemmar väljer att inte engagera sig.
    \item Ta fram en handlingsplan för att öka engagemanget på både grund- och
      masternivå.
    \item Eventuellt påbörja genomförandet av handlingsplanen.
  \end{FokusPunkt}

  \begin{FokusPunkt}{Internationalisering}{
    Ett stort problem som finns inom studentkåren är att få internationella
    studenter att känna sig delaktiga i verksamheten.  Bland
    Datateknologsektionens medlemmar är det en inte obetydlig andel
    internationella studenter och då vi har obligatorisk medlemsavgift är det
    vår plikt att även dessa medlemmars behov tillgodoses.
  }
    \item Tillgängliggöra mer av sektionens verksamhet för internationella
      studenter.
  \end{FokusPunkt}

  \begin{FokusPunkt}{Festkulturen}{
    Ibland är det fest. Ibland är den på data. Vi vill att den ska vara bra och
    trevlig för alla berörda parter.
  }
    \item Utvärdera existerande rutiner och riktlinjer gällande hur vi ska bete
      oss i våra lokaler för att skapa mer glädje än problem.
    \item Eventuellt förbättra dessa effter behov.
    \item Arbeta tillsammans med andra förtroendevalda på sektionen för att
      skapa en god festkultur.
  \end{FokusPunkt}

  \begin{FokusPunkt}{Informationskanaler}{
    Sektionsstyrelsen 17/18 har upplevt en del svårigheter i att nå ut till
    sektionens medlemmar. Ingen av de  informationskanaler vi har har i
    dagsläget når effektivt ut till alla eller ens en majoritet av våra
    medlemmar. Detta gör det svårt att nå ut med viktig information samt
    minskar transparensen i vårt arbete.
  }
    \item Se över hur vi förmedlar information till våra medlemmar och hur det
      kan förbättras.
    \item Eventuell genomföra förändringar samt utvärdera deras effekt.
  \end{FokusPunkt}

  \begin{FokusPunkt}{Hållbarhet}{
    Kårens vision för miljöarbete lyder som följer \textnormal{''Alla medlemmar
    ska kunna känna sig stolta över att Chalmers Studentkår agerar ansvarsfullt
    i miljöfrågor.''}. Sektionsstyrelsen 17/18 anser att det finns stor
    möjlighet till förbättring inom miljöarbete och hållbar utveckling på
    sektionen.
  }
    \item Arbeta med att minska användandet av engångsartiklar på sektionen.
    \item Arbeta för mer källsortering på sektionen.
    \item Utvärdera andra sätt att minska sektionens miljö och klimatpåverkan.
  \end{FokusPunkt}

  \begin{FokusPunkt}{Omstrukturering i bokföring och rapportering}{
    Sektionsstyrelsen 18/19 upplever att förtroendevalda har upplevt förvirring
    vid rapportering av sin verksamhet samt ekonomi, vilket medfört att många
    inte blivit ansvarsfria. Detta tros bero på otydliga rutiner och dålig
    kontinuitet.
  }
    \item Utvärdera och eventuellt förbättra ovanstående under året.
  \end{FokusPunkt}

  \begin{FokusPunkt}{Kontinuerligt arbete}
    \item Bevaka studenternas intressen vid utvecklingen av D-programmet och
      arbeta för ett förbättrat studentinflytande.
    \item Upprätthålla en trivsam studiemiljö och värna om studenternas hälsa.
    \item Hålla god och regelbunden kontakt med programledningen, kårledningen
      och våra utbildningsområdesrepresentanter.
    \item Arbeta med kontinuitet och ekonomiska rutiner inom sektionens
      kommittéer.
    \item Värna om Datateknologsektionens varumärke.
    \item Informera om och verka för större utnyttjande av Idéfonden.
    \item Informera om och verka för större utnyttjande av kårens äskningsfond.
    \item Arbeta transparent mot sektionens medlemmar.
    \item Arbeta med GDPR.
    \item Arbeta för jämlikhet på sektionen.
  \end{FokusPunkt}
\end{document}

